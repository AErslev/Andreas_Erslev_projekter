\documentclass[]{article}
\usepackage[utf8]{inputenc}
\usepackage{amsfonts,amsmath,amssymb,stackengine,xcolor}
%opening
\title{Aarhus Å}
\author{Andreas S. Erslev}
\renewcommand{\abstractname}{Aarhus Å}
\begin{document}
	
	\begin{center}
		\Large\textbf{Aarhus Å}
	\end{center}
	
	\begin{center}
		\large\textbf{Andreas S. Erslev}
	\end{center}
	
	Jeg kom gående igennem en gyde, midt I Aarhus by. Der lugtede af pis.
	Klokken var efterhånden blevet mange og sommerens dunkle lys, havde stille
	lagt sig, i byens ellers så oplyste gader. Grafittien på væggene vekslede mellem
	en dårligt Tegnet penis, til storslået kunstværker, nogle gange med en penis på.
	Min mor havde beordret mig at tage en ekstra trøje på. Selvom min alder ellers
	er oppe i tyverne. Set i retroperspektiv, er jeg glad for min mors insisteren.
	Den kolde kulde giver anledning til kuldegysninger og gåsehud.
	\newline
	\newline
	Selvom mørket havde lagt sig, kulden gav kuldskær og gadelygternes lys sad
	lige i øjenene, var jeg glad for den udendørs vandring, jeg havde begivet mig ud
	på. Mine tanker racede afsted, med en vis hastighed. Det var begyndt at ske
	oftere. Tanker der gerne gør mig istand til at forstille at guds plan for mig, må
	være ligeså storartet som Einstein eller Jesus. Tankerne var og er stadig, ikke
	altid direkte og rationelle. Manisk.
	\newline
	\newline
	Jeg havde været hjemme hos en af mine bekendte. Kløgtige Mads, som vi
	plejede at kalde ham. Lidt ironisk, da Mads’ kløgtighed ikke kunne måles med
	resten af slænget. Dette er jeg ikke sikker på at han ved. Det er sikkert også
	for det bedste.
	\newline
	\newline
	Hans lejlighed var noget sølle, med et badeværelse ved siden af køkkenet. Begge
	rum med dårlig udsugning. Den hærlige lugt af kattofler i ovnen, blandet med
	den hæslige lugt fra de udskylde oppusted kattofler. Han bruger sit værelse, på
	ca. ni kvadratmeter (otte, men i den høje ende, hvis man overvejer kommatalende). Der er lige plads til hans skærmfetish, tre på skrivebordet, og et TV på en lille reol, en reol til, en seng, enmands da han er single, en kommode og en sofa en ret god sofa. Hvordan kløgtige Mads har skabt plads til alt dette, må kun guderne vide. Eller måske er det muligt, at alt den tetris han valgte at spille i folkeskolen, endelig har givet ham en fordel i livet.
	\newline
	\newline
	Byvandringen har ledt mig ned til Arhus’ Å. Dette er punktet, hvori forståelsen
	for, hvorfor denne nat sidder indprintet i mit hoved. En af de få gange, hvor
	man kan være sikker på at printeren altid virker. Der lå en i åen. Spørgsmålet
	var, og er på sin vis stadig i dag, hvad jeg kunne gøre ved netop dette. Mens
	jeg ringer et-et-to, overvejer jeg hvor smart det ville være, både æstetisk og
	praktisk, hvis man kunne have en paraply og en fiskestang i én. Så kunne jeg
	blot fiske vedkommende op. Hvis jeg da ellers havde en paraply. Hvis jeg da
	havde et mod til døds fiskeri.
	\newline
	\newline
	Der kom en kranbil. En forvokset fiskesang, hvis du spørger mig. Der dog også
	kan bruges, i terræn der ikke er vand. Dog manglende en paraply. De hiver
	vedkommende op af vandet, mens en rar mand kaldet Niels, står og udspørger mig om diverse ting. Det er rart. Det er ikke så ofte folk spørger ind til min dag.
	Så udluftningen af dagens forløb, bliver lidt mere detalieret. Men det virker som
	om Niels er glad for det. Sikke en hærlig mand.
	\newline
	\newline
	Det er en kvinde. Et kønt stykke kvinde køn. Hvis man ser bort fra vandets hærdning og dets ellers så renlige affald. Hendes hud var smuk og blej.
	Formentlig grundet den grundige fermatering vandet havde udøvet. Sort hår.
	Knald sort. En fantastisk kontrast til hendes smukke hud. Det var svært at se
	ansigtet, men det lignede at hun havde striper, en form for ar, ned over ansigtet
	over det hele. Et meget gustent syn. Tøjet sagde mig ikke meget, men jeg er jo
	hellere ikke modeansvarlig. Det er trist man ikke kan se hendes øjne.
	\newline
	\newline
	Det er nu, efter den grundige analyse af liget, at hammeren falder. Chocket
	slår til. Jeg besvimer. Og vågner. På et hospital. Sikker noget. Sikke en aften.
	En oplevelse for livet, bør porienteres.
	
	
\end{document}
