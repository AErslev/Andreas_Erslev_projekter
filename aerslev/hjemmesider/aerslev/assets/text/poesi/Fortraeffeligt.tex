\documentclass[]{article}
\usepackage[utf8]{inputenc}
\usepackage{amsfonts,amsmath,amssymb,stackengine,xcolor}
%opening
\title{Fortræffeligt}
\author{Andreas S. Erslev}
\renewcommand{\abstractname}{Fortræffeligt}
\begin{document}
	
	\begin{center}
		\Large\textbf{Fortræffeligt}
	\end{center}
	
	\begin{center}
		\large\textbf{Andreas S. Erslev}
	\end{center}
	
	Det var en tidlig tirsdag morgen, og jeg tømte tarmene, på et fortræffeligt, funktionsdygtigt, dog forgængeligt, toilet, over en kop kaffe.
	\newline
	\newline
	Morgensmøgssmagen hang stadig fortræffeligt i kæften, selvom kaffen desperat forsøgte at overtage dette privilegium.
	\newline
	\newline
	Gulvfliserne var brækgule
	\newline
	\newline
	Badeforhænget var malet fabelagtigt, med motiver af blomster, dyr, vand, generel natur, der dannede et ansigt.
	\newline
	\newline
	Simon Peder
	\newline
	\newline
	Jeg bør nok tage et bad
	\newline
	\newline
	Toiletsædet var varmt. Dejligt. Min største præstation. Uden for er der også varmt. Men ik på samme måde. Alt venter. Realiteten lever ikke videre mens man skider. fortræffeligt.
	\newline
	\newline
	Realiteten stinker
	\newline
	\newline
	Væggene var smukt malet, med en eventyrligt neongrøn farve. Kald mig patetisk, men det var næste panteisme og side der og dyde den ældste dyd. Naturlov nr. 1. Livets cirkel. Alt ender som affald. Alt ender som afføring.
	\newline
	\newline
	Undtagen afføring. Det bliver til gødning. Fortræffelig ironi, hva?
	
\end{document}
