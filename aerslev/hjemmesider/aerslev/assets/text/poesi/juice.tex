\documentclass[]{article}
\usepackage[utf8]{inputenc}
\usepackage{amsfonts,amsmath,amssymb,stackengine,xcolor}
%opening
\title{En tinder samtale omkring juice}
\author{Andreas S. Erslev}
\renewcommand{\abstractname}{En tinder samtale omkring juice}
\begin{document}
	
	\begin{center}
		\Large\textbf{En tinder samtale omkring juice}
	\end{center}
	
	\begin{center}
		\large\textbf{Andreas S. Erslev}
	\end{center}
	
	Jeg fik engang spørgsmålet: “Synes du også at det er sådan ifht. Kærligheds livet?” efter en længere samtale, på tinder, vedrørende juice, hvilke typer der er favoritter, og hvordan man til tider ønsker en anden juice, frem for sin sædvanlige.
	\newline
	\newline
	Nej. Dog også jo. lød mit svar.
	\newline
	\newline
	Kærlighed er en mystisk følelse.
	\newline
	\newline
	Man kan have en specifik kærlighed, til en specifik person, men pludselig kan den specifikke følelse ændre sig.
	\newline
	\newline
	Derfor, jo, på sin vis.
	\newline
	\newline
	Men, jeg tror på, at man altid kan finde en, hvor selvom
	\newline
	\newline
	kærligheden forsvinder, ved man at den kan bluse op igen.
	\newline
	\newline
	“Jeg kan følge dig rigtig godt, det er vise ord og tanker” siger hun, mens jeg klapper mig selv på skulderen.
	\newline
	\newline
	Det er ligesom juice, fortsætter jeg.
	\newline
	\newline
	Nogen gange bliver man træt af appelsin (min præfereret juice). Men man bliver ved med at drikke den, da man ved, at glæden igen vil komme, ved den fortylledende smag.
	\newline
	\newline
	Nogen gange kræver den en lille pause. Nogen gange kræver det en ny drikke, som mælk. Nogen gange, en anden smag.
	\newline
	\newline
	Men det skal der også være plads til, mener jeg.
	\newline
	\newline
	“Nogle gange skal man prøve en anden juice i noget tid” spørger det kløgtige tinder match.
	\newline
	\newline
	Kommer an på hvilken kvalitet juicen har.
	\newline
	\newline
	“Haha, det er en god pointe”
	\newline
	\newline
	Nogen gange er den så skidt, at man skal smage noget nyt. Nogen gange er den så god, at selvom den kan blive kedelig og vag og hverdagsagtig, er der stadig noget godt ved den.
	\newline
	\newline
	Juice er en fantastisk analogi.
	\newline
	\newline
	“Hvad mener du om sagen”, spørger jeg ihærdigt efter sådan et spørgsmål, har fået mine tanker på geled.
	\newline
	\newline
	“Mine tanker og vise ord differentiere efter, hvilket emne der er tale om. Men ifht. juice/kærlighed tror jeg, at vi er ret enige”
	\newline
	\newline
	Igen, klapper jeg mig på skulderen, denne gang dog den anden.
	\newline
	\newline
	“Skal man tage hensyn til juicen” forsætter hun “eller man skal man kun handle egoistisk?”
	\newline
	\newline
	“Hvilken god analogi” siger hun, og denne gang bliver begge skuldre befamlet.
	\newline
	\newline
	“Ville have skrævet “virkelig god analogi””. Denne gang er skuldrene ikke nok, og et smil overtager det egocentriske behov.
	\newline
	\newline
	“Men jeg har ikke fået min eftermiddagslur idag”. En kløgtig reference til hendes egen profil, hvor der står: “Jeg går meget op i at sove eftermiddagslur”.
	\newline
	\newline
	Sikke en samtale man kan få sig, efter spørgsmålet “Vil du drikke en brick juice med mig?” som min beskrivelse består af.
	\newline
	\newline
	Tinder er et sted, med “Duck Faces”, bryster og til tider næseringe.
	\newline
	\newline
	Tinder er åbenbart også et filosofisk diskussionsforum.
	
\end{document}
