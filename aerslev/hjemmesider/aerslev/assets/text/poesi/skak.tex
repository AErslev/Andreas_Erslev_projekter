\documentclass[]{article}
\usepackage[utf8]{inputenc}
\usepackage{amsfonts,amsmath,amssymb,stackengine,xcolor}
%opening
\title{Et Godt Spil Skak}
\author{Andreas S. Erslev}
\renewcommand{\abstractname}{Et Godt Spil Skak}
\begin{document}
	
	\begin{center}
		\Large\textbf{Et God Spil Skak}
	\end{center}
	
	\begin{center}
		\large\textbf{Andreas S. Erslev}
	\end{center}
	
	Jeg sidder på en café og spiller et godt slag skak,
	Tænker de tanker omkring de ord der aldrig blev sagt,
	Fortabt, forlagt, sunket i dybden i spillet om magt,
	Mens min motorik motiveres af musikkens takt,
	\newline
	\newline
	Mens størstedelen af selskabets samtale synes at synke syd,
	Ned i nærheden af nærmelsen af normalitetens fryd,
	Fortabt, forlagt, for satan, forsvind,
	I dybet i de ubevidste dele af mit sind,
	\newline
	\newline
	Fantasien om det fantastiske forsvinder falmende,
	Kaffen i kruset kradser kedeligt og kvalmende,
	Kaffen køles langsomt, til en temperatur lig
	273,15 °K (grader kelvin)
	\newline
	\newline
	Bussen melder sin ankomst, gennem rejseplanen,
	Tilbage sidder jeg med min ven, en chatreuse, for fan,
	Spillet om magt, skak, forsætter nådesløst, man bliver træt,
	Mens brikkerne, vi ikke normalt har, skifter plads på det ternede bræt,
	\newline
	\newline
	Og med et sker det, trækket bliver trukket, og der er skakmat,
	Min homie kigger op, fnyser utilfreds og siger: det noget være pjat,
	Vi går op til baren og køber en øl, på køl,
	Fantasien om det fantastiske har sin opstand efter det køb,
	\newline
	\newline
	For andre er dagen slut, for os er aftenen lige begyndt,
	Og stoppe nu, ja, det må man fandme sig er synd,
	Så en øl bliver til ti, og timerne trisser stille imod i morgen,
	Og glæden i dag, bliver trukket fra i morgen, i form af tømmermand-sorgen,
	\newline
	\newline
	Intet er mere interessant, end spillet vi spillede.
	
\end{document}
