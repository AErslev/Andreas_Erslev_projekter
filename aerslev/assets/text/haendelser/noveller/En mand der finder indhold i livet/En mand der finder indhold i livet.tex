\documentclass[]{article}
\usepackage[utf8]{inputenc}
\usepackage{amsfonts,amsmath,amssymb,stackengine,xcolor}
%opening
\title{I}
\author{Andreas S. Erslev}
\renewcommand{\abstractname}{Vielse}
\begin{document}

\begin{center}
	\Large\textbf{En mand uden meget indhold i livet}
\end{center}
\begin{center}
	\large\textbf{Andreas S. Erslev}
\end{center}

\begin{center}
	\large\textbf{1. Aarhus’ Å}
\end{center}

Jeg kom gaående igennem en gyde, midt I Aarhus by. Der lugtede af pis. Klokken var efterhånden blevet mange og sommerens dunkle lys, havde stille lagt sig, i byens ellers så oplyste gader. Grafittien på væggene vekslede mellem en dårligt Tegnet penis, til storslået kunstværker, nogle gange med en penis på. Min mor havde beordret mig at tage en ekstra trøje på. Selvom min alder ellers er oppe i tyverne. Set i retroperspektiv, er jeg glad for min mors insisteren. Den kolde kulde giver anledning til kuldegysninger og gåsehud.
\\ \\
Selvom mørket havde lagt sig, kulden gav kuldskær og gadelygternes lys sad lige i øjenene, var jeg glad for den udendørs vandring, jeg havde begivet mig ud på. Mine tanker racede afsted, med en vis hastighed. Det var begyndt at ske oftere. Tanker der gerne gør mig istand til at forstille at guds plan for mig, må være ligeså storartet som Sokrates eller Jesus. Tankerne var og er stadig, ikke altid direkte og rationelle.
\\ \\
Jeg havde været hjemme hos en af mine bekendte. Kløgtige Mads, som
vi plejede at kalde ham. Ironisk, da Mads’ kløgtighed ikke kunne måles med resten af slænget. Dette er jeg ikke sikker på at han ved. Det er sikkert også for det bedste. Der havde været øl. Nok til en skabt mangle klips i klipsemaskinen. 
\\ \\
Hans lejlighed var noget sølle, med et badeværelse ved siden af køkkenet. Begge rum med dårlig udsugning. Den hærlige lugt af kattofler i ovnen, blandet med den hæslige lugt fra de udskylde oppusted kattofler. Han bruger sit værelse, på ca. ni kvadratmeter (otte, men i den høje ende, hvis man overvejer kommatalende). Der er lige plads til hans skærmfetish, tre på skrivebordet, og et TV på en lille reol, en reol til, en seng, enmands, en kommode og en sofa. En ret god sofa. Hvordan kløgtige Mads har skabt plads til alt dette, må kun guderne vide. Eller måske er det muligt, at alt den tetris han valgte at spille i folkeskolen, endelig har givet ham en fordel i livet.
\\ \\
Byvandringen har ledt mig ned til Arhus’å. Dette er punktet, hvori forståelsen for, hvorfor denne nat sidder indprintet i mit hoved. Selvom der er  svært, når den mangler farvepatroner. Der lå en i åen. Spørgsmålet var, og er på sin vis stadig i dag, hvad jeg kunne gøre ved netop dette. Mens jeg ringer et-et-to, overvejer jeg hvor smart det ville være, både æstetisk og praktisk, hvis man kunne have paraply og fiskestangs fusion. Så kunne jeg blot fiske vedkommende op. Hvis jeg da ellers havde en paraply. Hvis jeg da havde et mod til døds fiskeri.
\\ \\
Der kom en kranbil. En forvokset multi terræns fiskesang, hvis du spørger mig. Dog med mangle på paraply. De hiver vedkommende op af vandet, mens en rar mand kaldet Niels, står og udspørger mig om diverse ting. Det er rart. Det er ikke så ofte folk spørger ind til min dag. Så udluftningen af dagens forløb, bliver lidt mere detalieret. Men det virker som om Niels er glad for det. Sikke en hærlig mand.
\\ \\
Det er en kvinde. Et kønt stykke kvinde køn. Hvis man ser bort fra vandets hærdning og dets ellers så renlige affald. Hendes hud var smuk og blej. Formentlig grundet den grundige fermatering vandet havde udøvet. Sort hår. Knald sort. En fantastisk kontrast til hendes smukke hud. Det var svært at se ansigtet, men det lignede at hun havde striper, en form for ar, ned over hele ansigtet. Et meget gustent syn. Tøjet sagde mig ikke meget, men jeg er jo hellere ikke modeansvarlig. Det er trist man ikke kunne se hendes øjne. Jeg forstiller mig grønne.
\\ \\
Det er nu, efter den grundige analyse af liget, at hammeren falder. Chocket slår til. Jeg besvimer. Og vågner. På et hospital. Sikker noget. Sikke en aften. En oplevelse for livet, bør porienteres.

\begin{center}
	\large\textbf{2. Uden bukser på}
\end{center}

Jeg har hjernerystelse. Jeg ligger på hospitaltet. Uden busker på. Jeg vågner simpelthen op, i en fremmede seng uden bukser på, mens en flot pige kommer med morgenmad, og spørger om jeg har sovet godt. Det er sku aldrig sket for mig før. Jeg er ør i hovedet, og formulere et elendigt svar. "Jaja, helt sikkert der, det var sku ok, jeg var helt væk." Fuck mig. Hun er høffelig. Det følger vel med til jobbet. Hun går. Jeg putter pænt min serviet fast i blusen, i tilfælde af at jeg skulle spilde. Jeg spiser roligt min havregrød. Bedere end forventet. Jeg drikker glasset med juice i en tår. Ja, tørsten var sku stor. Sådan er det, når man stadig samler hjerne celler op fra jorden.
\\ \\
Rummet var sterilt. Hvide væge. Hvide senge. Vidt over alt. Sågar en hvid sengekammerat. Der er dog et billede på vægen. Det er af en strand. En hvd strand. Med hvide mennesker der går på den. Blåt hav med hvid refleksion. Hvide skyer. Der kom en mand i hvid kittel og sagde "Du må godt gå nu!" med lidt irretation i stemmen. Så jeg tog mit tøj på. Der kom lidt farver på værelset i et kort øjeblik, før jeg gik ud af døren.

\begin{center}
	\large\textbf{3. Mine bedste venner}
\end{center}

Det blev morgen. Eller, klokken slå fjorten. Jeg vågnede. Tog min morgenkåbe på og gik ud i køkkenet. Lavede 2 spejlæg. Med timina, salt og peber. Jeg gik ind på mit værelse. I mørke. Mørklægningsgardinerne blev ikke trukket fra. Solen havde ingen plads på værelset. Jeg tændte op for et afsnit "Friends". Sæson 3, Episode 3. "The One With The Jam". I en tidligere episode, havde Richard og Monica slået op. Grunden til dette var, at Richard ikke ville have børn, mens Monica gerne ville. De så derfor ingen fremtid i forholdet. I episoden lige før Episode 3, prøver Ross, at få slænget til at gøre sig klar, til en vigtig begivenhed på museet, hvor han arbejder. Både Chandler og Joey drikker fra et glass fyldt med fedt, i troen om at det er cider. Monica er ikke særlig glad for dette. Gad vide, hvad hun skulle bruge et glas fedt til? Sikkert til at stege frikadeller.
\\ \\
Det hele starter med, at Chandler sidder og læser en bog. Vi for ikke oplyst, hvad bogen handler om. Der kommer nogle knirke lyder, fra en seng, inde fra Joey's værelse. Her tænker man straks, at han elsker med en kvinde. Vi hører et skrig, og finder ud af at Joey hoppede i sengen, og da faldt ned. Chandler siger "See, Joe, that's why your parents told you not to jump on the bed." En fantastisk start. Den holder virklig ens interesse, med et smil på læben.
\\ \\
Vi bliver intruduceret til afsnittet, ved at Monica laver "Jam" fordi, hun er deprimeret over Richard. Hele slænget kommer ind. Undtage Phoebe. Joey er vild med "Jam". Han smager på lidt direkte fra gryden, men det er for varmt. han ender med at spytte det tilbage. Ha, klassisk Joey. Joeys arm også kommet til skade med sin arm, da han faldt ned fra sengen. Den er i slynge
\\ \\
I den næste scene, ser vi Phoebe blive forfuldt af en mand. Men han tror det er hendes søster, Ursula, som han datede i lidt tid. Han stalker altså den forkeret søster. Phoebe tilgiver ham. Hun giver rådet "You are not a witch, you are just an avarege student" til stalkeren. Hun ender med at invitere ham på kaffe.
\\ \\
Rahcel og Ross kommer op i den lillae lejlighed, hvor de opdager at de er alene. De "hygger" lidt. Men Chandler kommer ind. Det stopper brat. Han har et problem med sin kæreste "Janice". Jeg vil ikke gå formeget i dybden med problemet. Det må være en lille teaser til dig som læser, så du selv kan se episoden, og stadig have nogle overraskelser. Dem skal jeg nok lave lidt flere af ;)
\\ \\
Vi befinder os nu på caféen "Central Perk" hvor joey sidder op putter, he, lidt for meget, kan man vel sige, syltetøj på en scone. Eller en anden type bolle, i hvert fald. Hele slænget er det, ud over Phoebe og Monica. Phoebe kommer dog ind, og slutter sig til gruppen. Hun fortæller om sit møde med sin stalker "Malcome". Joey sige, he, han er sku sjov, han siger "You talked to him?" mens hans mund er helt fyldt med bolle. Der er krummer ud over det hele. Monica joiner da festen, så nu er de alle samlet. Chandler stiller spørgsmålet "Det girl from det xerox place, buck naked, or a tub big of jam?" og Joey, sjov som han er, siger "Put your hands together" Haha, den for mig vær eneste gang, jeg ser afsnittet. Monica fortæller om sin nye plan, for at komme over Richard. Babyer. Via en sæd donor.
\\ \\
Chandler oplever de samme problemer, men har nu en løsning fra Ross. Men igen, se du det selv. Så er det lidt mere spændende. Phoebe ses hvor hun hjælper stalkeren "Malcome". Men det må du også selv se, jeg vil ikke røbe for meget
\\ \\
Monica mener hun har fundet sin perfekte sperm donor. De andre taler imod idéen. Men hun forsvar det. Joey laver "Jam crackers". Han er altså hyle morsom. Hun finder Joey, som sæd donor. Phoebe foræller om hendes møde med Malcome, og så kommer en scene med Chandlers problemer, og igen Phobe's. Igen, det må du selv se, din bavian. Det er et godt afsnit.
\\ \\
Moica har valgt sin donor. Hun snakker lidt med Joey omkring det. Han fortæller om hvordan han ville have set hendes fremtid, yderligere med hendes fremtidige mand. Han forstiller sig at de har skilt, hvor der står "We don't swm in your toilet, so don't pee in our pool." Ha, som om det ville stoppe mig. Han fortæeller om deres fælles børn. En dreng og to piger. Samtalen for Monica til at skifte mening. Men bliver lidt ked af det, og som den gode person Joey er, trøster han hende. Sidste scene er omkring Joey, der sidder og snakker om at han der ikke er nogen der har været intresseret i hans sæd. (det vill jeg være, hvis jeg var en kvinde, hehe) Han spiser syltetøj, og for det ud i hele ansigtet. Ross kommer hjem, og Rachel er sur på ham. Det er grundet Chandler situationen, som jeg afslører intet af. Afsnittet er slut.

\begin{center}
	\large\textbf{1. Aarhus' Århusianer Vaner}
\end{center}

Jeg sidder i min sofa. NetFlix har spurgt om jeg stadig ser venner. Det gør jeg ikke. Skærmen går i dvale, og bliver sort. Jeg kigger i det sorte spejl og ser mig selv. Jeg klør mit i skridtet. Jeg finder min mobil frem for at ornanere. Efter jeg er færdig, låser jeg skærmen. Igen, kan jeg se mig selv. Jeg skammer mig. Jeg har ikke lavet andet end at spise æg, se venner og onanere de sidste par dage. Måske jeg skulle se at komme ud af huset. Det er jo sol og sommer.
\\ \\
Jeg er solallergiker. Det vil sige, at i skarpt lys, så som stærk sol, nyser jeg. Denne sommer er det gået helt amok. Nys efter nys efter nys. Min cykel er punkteret. Øv. Det vidste jeg godt. Det gider jeg ikke fikse. Jeg går ned mod bussen. Jeg har glemt mit rejsekort. Øv. Jeg må gå. Jeg kan ikke lide at udnytte den offentlige transport. Det er jo nærmset tyveri at køre uden billet. Gåturen er lang. Eller, for mig er den lang. Den tager godt og vel femogfyrre minutter for de fleste. For mig tager det tres. Jeg er et reflektivt menneske, og elsker at kigge på omverdenen, mens jeg bevæger mig igennem dem.
\\ \\
Jeg går ad store veje, små veje, ved cykelstiger, universitets parken og Ø gade kvarteret. De store veje er et godt blik på Aarhus. Altid stress. Alle skal et eller andet sted hen, hurtigt. På de store veje, har alle ideologien om, at de er de vigtigste og eneste på vejen. De store veje er kun gode, da man må fokusere viljefast på musikken. Musikken leder til tanker. Tanker lede til god tidsfordriv. Jeg høre den samme sang på repeat.
\\ \\
Små veje og cykelstiger er lidt bedre. jeg nyder at se folk i deres hverdag. Fundere over deres liv. prøve at forud se, om de er sent på den, på vej hjem til en fodbold kamp eller skal til fest. Der er altid nogle der skal noget, men modsat de store veje, er folk her observante på hindanden. På dette tidspunkt på dagen, er det dog mest folk der tager det meget afslappet. Det er en rar atmosfære at være i. Det samme er der i universitetsparken. Der er lidt ude og drikke øl. Nogen har en højtaler med. Andre spiller spillet "kævle". 
\\ \\
Det handler om at stå to hold på være sin side. Typisk 3 meter fra midten. I midten står der en kævle. Det gælder nu om at vælte kævlen. Mand har et forsøg. Man kaster med en sko. Rammer man kævlen, så den vælter, skal man drikke af sin øl, der selvfølgelig fra start er fuld. Her gælder det om for det andet hold, at at hente skoen, samt sætte kævlen på plads igen. Nå dette er gjort, skal det andet hold stoppe med at drikke øl. Når alle af holdets øl er tømt, vinder holdet. Sikke et spil.
\\ \\
I parken spilles mange andre drukspil. Det mest velkendte ville vel være "øl bowling". Det er så kendt, at jeg går ud fra at jeg ikke behøves at forklare reglerne. Ellers så brug google. Det er altid rart at vandre i en park. Specielt om sommeren. Specielt denne park. Med alle de studenterrelatered traditioner, så som kapsejladsen. Ja, det er altid rart, selv om vinteren, at vandre i et naturområde med liv, midt i en stressende by.
\\ \\
Ø-gade kvarteret er blot et hyggeligt område. Her bor så mange friske unge mennesker. Lejlighederne er gamle. Rustikke. Forskellige. De er unikke. Alle jeg har kendt, der boede her, har været ekstatiske over at bo her. Og man forstår godt hvorfor. Som jeg sagde om de stille veje, giver man plads til hindanden. Dette er et helt område, hvor folk giver plads til hindanden. Mangfoldighed.
\\ \\
Jeg når ned i latiner kvarteret. Min disination. Mit favorit område. Det er det sted, i byen, hvor jeg føler mig mest hjemme. Mødestedet for mig og mine bekendte. Mødestedet, hvor man drikker kaffe om dagen, og øl om aftenen. Ja, det er her jeg begiver mig hen. Grunden er kaffe. Jeg ville ønske jeg havde en i mit liv, der ville drikke en med mig, men dette må blive en anden gang. Denne gang er det mig, kaffen og cigaretten. "En kop kaffe, gerne så sort som sjælen selv" siger jeg selvsikkert, mens jeg fyre op for et søm. Et søm, der stille bankes i kisten.

\begin{center}
	\large\textbf{1. Aarhus' Ånd}
\end{center}

Jeg sidder nu på min favorit café. Jeg kan lide den, fordi musikken er god, dog i sådan en volummen, der gør at man stadig kan høre hvad der sker. Jeg sidder ved et bord. Et tomands bord, hvor der faktisk godt kan sidde fem. Jeg skodder min cigaret. Tænder en ny en. Inhalation. Ekshalation. Inhalation. Ekshalation. Sug ind. Pust ud. Det kunne næsten lyde som om, en jordmorder var ved at lære mig at ryge.
\\ \\
Mens jeg sidder og lytter til samtalerne der er omkring mig, da de er meget mere indholdsrige end mit liv. Lidt mere spændende end binging Friends. Ved sidden af mig, er der en ung pige. Hun har problemer med sit sex liv. Hendes dating partner fra tinder holder for længe i sengen. Hun fortæller, at det varer mindst tredve minutter. Ofte længere. Hun keder sig under sex og hendes nedre dele er begyndt at gøre ondt. Rigtig ondt. Hendes veninde forslår, om det kan være at det er fordi han ikke tænder nok på hende. De griner begge to. Deres næste samtale er ikke spændende, så jeg bevæger satelitten til nye lydbølger. 
\\ \\
Der er et andet spændende par. Deres samtale går meget langsomt. Ganskevist fordi de spiller skak. Dette gør samtalen, eller i hvert fald situationen, meget mere spændende. Selvom de ikke er særlig gode. Jeg har lyst til at hoppe ind, hele tiden, og give dem det oplagte træk. De kan jo tydeligvis ikke finde ud af det. De er slet ikke på mit niveau.
\\ \\
Efter at have skandet rummet med mine satelliter, må andres liv skuffe, da de giver mig mere lyst til at se Friends. Hold kæft hvor blev folk kedelige. Fuck dine studieproblemer. Jeg går på toilet. Toiletet har altid været et af mine ynglingssteder på caféen. På vægene er der ophængt gamle reklamer, med alt fra rengørningsmiddel, arnbitter og knorr bearnaise sovs. Selve toliet er rent. Ligeså vasken. Det sætter jeg pris på. Håndtørren irritere mig. Mine hænder bliver fandme aldrig tørrer. Men for at gøre det fair, så har jeg aldrig oplevet at nogen håndtørrer har virket ordentligt. Så det ser jeg bort fra.
\\ \\
Da jeg kommer tilbage fra toillet, sider der en kvinde ved mit bor. Bleg. Hendes hår var uglet. Hendes krop var bleg. Grønlig. Det lignede næsten at den var fermenterede. Havde en speciel udstråling. En gylden udstråling. En bleg gylden udståling. Hun var smuk. Uffateligt smuk. De samme striper i ansigtet som kvinden i åen. Hun sad og sang, med på teksterne musikken spillede. Hun sang højt. Men det var som om ingen lagde mærke til det. Det var som om ingen kunne høre det. Det var som om hun kun eksisterede i mine øjne.
\\ \\
Jeg gik op i baren for at købe en frisk kop kaffe. Da jeg kom tilbage og satte mig, kunne jeg se at der var to glas chartreuse. Med isterninger. Hun stoppede med at synge. "Hello Sir" sagde hun til mig. Med en sexet stemme. "My name is Meddyusa" forsatte hun. Hun tog en tår af sit glas chartreuse. Hun hentydede til, at jeg også skulle. Jeg var forundret. Facineret. Nysgerrig. Så jeg tog en tår. Det brændte og smagte ikke som det plejer at gøre.

\begin{center}
	\large\textbf{1. Aarhus' Åndsmenneske}
\end{center}

"Why are you here?" Jeg forsatte, mens jeg satte drinken ned. "What is you agenda?" Jeg tog en ciggeret og smed den i munden. "How rude of you not to present yourself." sagde hun med en vrissen tone i stemmen. Jeg kunne godt se, at jeg lige var lidt uhøffelig og skyndte mig straks med at svare. "Adam ." og jeg fortsatte "Would you know please awnser my question?" sagde jeg lidt irriteret. "You seem lonely. So I thought I would keep yor company." 
\\ \\
Hun begyndte at stille nogle personlige spørgsmål. "Have you erver found a woman, you could could and would keep in your life?" Et aggresivt spørsmål. Et spørgsmål jeg ikke helt vidste hvad jeg skulle svare på. Jeg havde aldrig haft det. Kun små flirts, hvor pigen gav op på mig. Det er det jeg må sige. "No, I haven't." sagde jeg med generthed i stemmen. "I guess you just never met the right one. One to keep you in check." sagde hun. Hun ville gerne høre mere om det. "I think I never found anyone that complement me. I guess nobody see's me as the right one." "Have you ever had sex?" she said, with a wink and a smile. "No. No i haven't. I'm fairly good at talking with girls, but it's never been more than a kiss."
\\ \\
Samtalen forsatte på denne måde et kort stykke tid. Men så begyndte hun at stille mig spørgsmål om mig. Om hvad jeg havde af hobbier. Hun lod mig tale. Når jeg taler reflektere jeg. Så det jeg ville sige, kørte altid over i noget andet. Hun lod mig bare tale. Dialogen blev til en monolog. Omkring mit liv. Omkring mine skader. Både fysisk og psykisk. Det var rart at have en som lyttede. Efter noget tid, spurgte hun, om vi ikke kunne gå et mere stille sted hen. Jeg tog hende med i parken. monologen forsatte. Lidt endnu. Så forslog hun, om vi ikke skulle tage hjem til mig. Jeg afviste. Jeg løj om at jeg blev nødt til at gå. Hun lagde op til sex. Min nervøsitet vedrørende sex var for stor.  Jeg kunne ikke håndtere det. Jeg forlod hende i parken. Hun virkede ikke glad. Hendes udstråling gik over, og blev mere aggresiv og fjendtligt. Hendes skær gik stille fra blåligt til blod rødt.

\begin{center}
	\large\textbf{1. Aarhus' Ånds Åndedrag, mit Åndedrætsbesvær}
\end{center}

Jeg tog en bus hjem. jeg var lidt i chock. Det var som om hun aldrig var blevet afvist før. Jeg følte mig træt i kroppen. Som om der havde været en besværgelse, der havde forsøgt at overtage min krop. Busturen er kun ti minutter lang. herefter skal man gå i fem-ti minutter. Alt afhængigt af om lyskrydset er grønt, og hvor hurtigt man går. Mine ben var stadig gummi. Så det tog faktisk et kvarter denne gang. Da jeg kom hjem, var min roomie ved at lave mad. Den gode paprika gryde, han laver mindst en gang om ugen.
\\ \\
"Så kunne du fandme være hjemme, hva' Adam. Det er jo lige før du ikke nåede maden." sagde han, med et smil på læben. "Nu skal du fandme høre hvad der skete for mig, makker." Jeg fortalte historien, over paprika gryden. *smask* *smask* "Og så tog vi i parken, og..." blev jeg ved. Men i kender jo historien, så jeg vil ikke gentage. "Kæft man det lyder sygt homie. Men hvorfor fanden tog du hende ikke med hjem? Det er jo gratis fisse lige der, homie. Det er sku da på tide, at du for parkeret bilen i garagen." 
\\ \\
Han havde en pointe. "Hvor meget bliver maden, Karsten?" sagde jeg, for at skifte emne. "Kun 13 kroner i dag du. Kødet var på min mors regning." sagde han med stolthed i stemmen. Det var også imponerende billigt. "Jeg har sendt på mobilepay" sagde jeg, mens jeg begyndte at tage af bordet. Den ene laver mad, den anden tager opvasken. Det er aftalen. Der var heldigvis ikke så meget, og så har vi en opvaske maskine. Så man skal ikke stå med tallerknerne. Det skulle jeg i min gamle lejlighed. Det var et heleved. Specielt når der havde været gæster.
\\ \\
Men jeg stod og dansede til den gentagende lyd af "New Drug" med Jinji Kikko, var det som om noget pludselig ånede mig i nakken. En kold ånde. Kort men skræmmende. "Det var sikkert bare lidt gennemtræk" tænkte jeg for mig selv. Det skete igen. Der var ingen døre eller vinduer åbne. Det var jo lidt koldt udenfor. Hvad fanden, tænkte jeg. Det skete ikke igen. Før morgenen efter.
\\ \\
jeg stod i badet. Hørte "Sexual Healing" af Marvin Gaye, på fuld skrald. Jeg sang med, godt nok i lav volummen, så Karsten ikke kunne høre det. Det skete igen. Mens jeg stod i badet. Denne gang voldsomt. Så voldsomt, at jeg næsten faldt. Jeg skyndte mig ud af badet. hele rummet var dampet. Jeg kan godt lide varme bade. Håndkldet blev smidt rundt på kroppen. Jeg var stadig våd på ryggen. Jeg gik ud af værelset, og fandt mine angst piller. Rivotril. Mand må tage tre om dagen. Jeg tog alle på samme tid. Jeg sad i ti minutter, klamret sammen i fosterstilling. Pillerne virkede endelig. Jeg kunne slappe af. Puha, sikke en oplevelse. 
\\ \\
Ånden blev ved. Jeg var så skræmt. Jeg kunne næsten ikke trække været. Jeg hørte døren smække. Karsten skulle på Uni. Det gjorde det ikke meget bedre at jeg var alene. jeg var lige ved at ringe til læge vagten. Man skal igennem læge vagten før man kan komme til psykiatrisk vudering. Da jeg havde tastede 5 ud af de 8 tal, bankede det på døren. Jeg gik ud og åbnede. Jeg kunne ikke lade vær. Det var som om noget styrede mig.

\begin{center}
	\large\textbf{1. Aarhus' Åndelig sang}
\end{center}

Der stod en kvinde. Et smukt eksemplar af en kvinde. Hun havde nogle træk ligende kvinden fra åen, dog var hun sin helt egen. Hun havde en del make-up på, men inden bagved kunne mand se en stribe. Det lignede næsten et ar. Hendes hår var langt, glat og blond. Det var så langt, at det gik helt ned til røven. Gad vide om hun nogensinde, er kommet til at skide lidt på sit hår? En finurlig lille tanke. Dem er vi glade for. Hendes krop var kurved. Hendes hud var mørk i det. Ikke sådan mulat/sort, men nærmere som en bleg krop som var lidt solbrændt. Hendes tøj var lidt slattent. Det passede ikke helt til hendes kropslige udstråling. Men som pointeret tidligere, har jeg ikke forstand på mode. Jeg vil derfor ikke kommentere mere på valget af tøj.
\\ \\
Mens jeg stod og beundrede det smukke syn, med elevatorblikke, der kørte op og ned, fra første sal, til tagterrassen og ned igen, stod hun og sang. En smuk sang. Det var dog lidt mærkeligt. Mit syn gik fra beundrende til små irriteret. "Hov, altså undskyld mig, men hvad er det du laver her, om jeg må spørge?" afbrød jeg hende, midt i hendes ellers fine sang. Hun så forvirret ud. Hun stod bare og kiggede lidt på mig. Så var det som om, hun kunne høre en stemme. Lidt alle det udtryk, en vært for, når vedkommendes producer fortæller dem noget i deres øresnegl.
\\ \\
"Må jeg få en kop kaffe?" spurgte hun "Jeg bor i området og har set dig gå ture. Du ser så sød ud, at jeg bare måtte have en snak med dig!" Lidt mærkeligt i mit syn, men jeg var aldrig blevet beundret før, i hvert fald ikke hvad jeg ved af, så lidt i chock gav jeg lov. Heldigvis er jeg lidt af en kaffe narkoman, så kaffen var allerede klar og varm. Jeg hældte en kop op til hende. spurgte om hendes navn. Det var Eva. Sikke et finurligt navn, når man nu overvejer mit. Adam og Eva. Begyndelsen på noget nyt. Begyndelsen på paradis. Jeg må bare holde hende fra æblet. "Vil du ikke med ind på mit værelse? Der er så mørkt i stuen" forslog jeg. "Det lyder bare godt." sagde hun næsten, før min sætning var færdig. Vi sad nu og snakkede lidt frem og tilbage. Men dialogen blev hurtigt til en monolog. Hun ville høre om mit kærligheds liv til at starte med. Det undrede mig.
\\ \\
Både pigen på caféen ville høre om mit kærlighedsliv, og nu denne skønne pige. Men siden hun havde beundret mig en smulle, giver det vel fin mening, at hun gerne vil høre om der kunne være en mulighed, for at skabe et forhold mellem os to. Så samtalen gik meget, som den på caféen. Monolisk. Fra kærlighed til hobby og hverdags liv. Jeg fik lov til bare at plapre løs. Det var faktisk lidt kedeligt. Så jeg beyndte at spørge ind til hende. "Hvordan er din hverdag så?" 
\\ \\
Hun så igen forvirret ud. Som om hun havde regnet med, at jeg bare ville plapre videre om mig selv. "Det er da kun høfligt, at spørge ind til vedkommendes liv." sagde jeg, med et smil på læben. "Ja, ja det har du ret i!" sagde hun, mens det igen lignede, at hun fik en besked i øret. "Jeg arbejder lige nu. I et sushi sted. Så kan jeg få billig sushi." forklarede hun. Hun fortalte videre. Ja sagde engang imellem "Ja" "Aha" "Spæendende" osv. Bare for at vise, at jeg altså hørte hvad hun sagde. Hun virkede mere naturlig og afslappet nu. Hun drak ikke af sin kaffe. "Hvordan kan det være, du ikke drikker af din kaffe?" spurgte jeg undrende. "jeg kan faktisk ikke lide kaffe. Det var bare en undskyldning for at snakke med dig." "Hvad med the? Kune det interessere?" "Ja, ja det ville være fint." Så jeg lavede noget the.
\\ \\
Men pludselig blev hun stiv. "Skal vi have sex?" sagde hun. Jeg var forbløffet. "Hvad fanden er det for et spørgsmål? Sex er en intim aktivitet. Det er altså ikke noget man bare skal gå rundt og tilbyde, bare fordi man har fået en kop the. Nej, du kan få en date med mig, det er det du kan få! sagde jeg med hidsighed i stemmen. Hovedsageligt fordi jeg var bange for sex. hvad nu hvis jeg ikke gjorde det ordentligt. Ville hun så aldrig ses med mig igen? Jeg kender slet ikke til den verden. Den er skræmmende! Hun så lidt skræmt ud. Men takkede ja. Hun ville gerne ses med mig igen. Jeg fik hendes facebook. Så kunne jeg bare skrive, når jeg fik tid. 

\begin{center}
	\large\textbf{1. Aarhus' Ånds Sammensvorne}
\end{center}

Der gik en lille uge, før vi så hindanden igen. Denne gang på min favorit cafe. Vi sås igen og igen. Nogle gange fortæk vi, at komme steder der var velkendte. Andre gange var det spændende at udforske nye steder. Vi gik mange ture. Langs åen, ned til vandet, ud forbi Aarhus Ø. op i botanisk have. Vi sad hjemme hos mig, til tider. Jeg fik dog aldrig lov til at se hvor hun boede. Efter vi havde setes en seks syv gange. Vi var alene hjemme hos mig. Jeg tog fat i mine nossere, og klæmte til. Billedeligt talt. Jeg kyssede hende. På hendes kind. Hun kyssede mig på munden. Det var magisk. Men der skete noget ved hendes udseende. Hele hendes kropsbygning ændrede sig. Mere kurvet. Hendes hør, begyndte at bølge og blive rødligt. Hendes hudfarve blev lidt mere vinter agtig. Hendes ansigt fik lidt mere æble kinder. "Jeg tror jeg elsker dig Adam" sagde hun. Så begyndte hun at græde, og løb ud af huset. 
\\ \\
Jeg kunne ikke få fat på hende. Jeg skrev og ringede firetyve-syv. Jeg bekymrede mig om hende. Hvad nu hvis der var sket hende noget. Der gik en lille uge. Så fik jeg en besked. "Undskyld" stod der. Jeg var forbløffet. Undskyld for hvad mon? Jeg skrev "Undskyld for hvad dog?" "Bare undskyld. Jeg er slet ikke den du tror." Hmmmmmmmm, hvad? "Kan vi mødes. Så kan du forklare. Uanset hvad det måtte være, er jeg sikker på at vi kan komme igennem det sammen." Hun godtog. Vi mødtes på caféen. "Hva' er det så der sker, min kære Eva." Hun var lige ved at begynde at fortælle, førr jeg blev nødt til at afbyde. "Forresten. Jeg har tænkt over det. Og jeg elsker altså også dig. Det troede jeg aldrig jeg ville få sagt. Men det gør jeg altså."
\\ \\
Hun begyndte at græde lidt igen. Med et smil på munden. "Adam. Jeg har været hjemsøgt af en demon. En ånd, der ville have hævn over dig. Hævn, fordi du øædelagde hendes søvn. En kvinde, der var druknet i åen. Hun siger, at hun er dronningen over sirener. Hun overtalte mig til at forføre dig, for at ødelægge dit hjerte. Men jeg kan ikke gøre det." sagde hun, med alvor i stemmen. "Aha. Syret. Okay. Jeg tror lige jeg skal processere det." Der gik lige to minutter, mens jeg sad i Grublerens positur. "Hvad med alt det du har fortalt mig? Er det så også løgn? Fordi det ville nok være det værste. Det ville betyde, at jeg ville have forelsket mig i den forkerte." sagde jeg med bekymring i stemmen. "Nej. Jeg har talt ud fra mig selv. Så vidt jeg ved, er grunden til hun har valgt mig, at jeg er det perfekte match til dig. Så jeg har skullet være ærlig, for at indfange dig. Men efter jeg udtrykkede min kærlighed til dig, har jeg ikke hørt fra hende siden. Det er som om hun er forsvundet."
\\ \\
Vi snakkede længe om det. Men det skulle ikke betyde noget. Hvis du hun havde sagt, var sandt, havde jeg forelsket mig i den rigtige. Også selvom hun måske havde fået lidt hjælp. Jeg kyssede hende igen. "Kom med hjem til mig!" Sagde hun. "Okay." sagde jeg. hun boede i en fin lille lejlighed. Det var tydeligt at hun boede selv, og brugte meget tid i lejligheden. hun var en total enspænder. Ligesom mig. Jeg havde godt haft det på fornemmelsen, men nu var jeg da helt sikker. "Jeg er klar nu." Sagde jeg. "Jeg er klar til at udtrykke min kærlighed fysisk!" "Sig nu bare forheleved sex Adam. Det er ikke vildere end en smule nydelse!" vi havde sex. Det var fantastisk. Hun er fantastisk.

\end{document}